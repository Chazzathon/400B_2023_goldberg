\documentclass[twocolumn]{aastex63}
\usepackage{graphicx} % Required for inserting images
\usepackage{natbib}


\begin{document}

\title{ASTR 400B Research Assignment 7 \\
Comparing the Kinematics of a Simulated Milky Way--Andromeda Merger Remnant to those of Observed Elliptical Galaxies}

\author[0000-0002-4901-7693]{Charlie Goldberg}

\affiliation{Steward Observatory \\
937 N. Cherry Ave \\
Tucson, AZ 85719, USA}

\date{26 April, 2023}

\keywords{Major Merger -- Elliptical Galaxy -- Velocity Dispersion -- Merger Remnant -- Rapid/Slow Rotator}

\begin{abstract}
    A simulation of the merger between the Milky Way and Andromeda galaxies is performed, with the remnant of the merger being analyzed for the characteristic properties of elliptical galaxies. Through this analysis, the progenitors of elliptical galaxies and their formation mechanisms are better understood. This study focuses upon two easily observable features in elliptical galaxies: their rotation speed and velocity dispersion curve. By characterizing these two features of the merger remnant from the simulation, the simulation products are directly compared to observed elliptical galaxies to determine if elliptical galaxies may form from the merger between two large spiral galaxies. The merger remnant is a fast rotator, which is inconsistent with most observed large ellipticals. However, the velocity dispersion of the remnant is largest in the core and decreases at larger radii from the center, which matches observations. Thus, mergers between two large spiral galaxies likely contribute to the formation of ellipticals, but do not fully explain their properties.
\end{abstract}

\section{Introduction}

When two galaxies approximately the size of the Milky Way and Andromeda collide in what is called a major merger, it is thought that an elliptical galaxy with a spherical shape and no spiral structure forms. To verify this, the kinematics of a merger remnant between a simulated Milky Way -- Andromeda major merger are studied and compared to observed elliptical galaxies.\\

A galaxy is an isolated system of stars that are gravitationally bound and orbit a common center of mass, but whose motions do not appear to abide by Newton's laws of gravitation and motion. Galaxies may take the shape of spirals, ellipses, amorphous blobs, and combinations of all three. Their size and shape change in the process of galaxy evolution, but it is not known precisely how. For example, it is believed that elliptical galaxies evolve from smaller galaxies, but their formation mechanism is not fully understood. Because elliptical galaxies make up over 10\% of all known galaxies (\cite{1996MNRAS.278.1025L}) and are thought to be the byproduct of galaxy mergers, their origins are of great importance to our understanding of galaxy evolution. By simulating galaxy mergers and studying their products, the formation of elliptical galaxies can be better understood.\\

It is currently known that elliptical galaxies often result from mergers between spiral galaxies (\cite{2006ApJ...650..791C}). In addition, the largest elliptical galaxies are thought to be the result of several mergers (\cite{2006ApJ...650..791C}). Observationally, elliptical galaxies appear to be dispersion supported, meaning that their components orbit in random directions as shown in Figure \ref{dispersion}. Because of this, ellipticals also have little bulk rotation as compared to spiral galaxies.\\

While the general formation mechanism of elliptical galaxies is understood, it is unclear why ellipticals have little rotation (\cite{1998A&AS..133..325G}) while their progenitors do. It is also not understood why smaller elliptical galaxies rotate more than their larger counterparts (\cite{2011MNRAS.414..888E}). In addition, it is unknown why many elliptical galaxies do not appear to be dark matter dominated, despite their progenitors being heavily dominated by dark matter (\cite{2003Sci...301.1696R}). Finally, it is unclear if ellipticals form from individual mergers of large galaxies, or if they are the result of many small mergers and some major mergers.

\begin{figure}
    \centering
    \plotone{dispersion.jpg}
    \caption{Image of NGC 3379 showing the relative velocities of planetary nebulae within the galaxy. The motion of the nebulae toward and away from the observer appears random, indicating the galaxy has little or no rotation. Figure from \cite{2003Sci...301.1696R}.}
    \label{dispersion}
\end{figure}

\section{This Project\label{proposal}}

The first question this study will address is: is the Milky Way -- Andromeda merger remnant a fast or slow rotator? A slow rotator is defined as a galaxy where the average circular velocity of particles is significantly less than the velocity dispersion, or $\frac{V}{\sigma}<0.6$ (\cite{sparke2007galaxies}), where the velocity dispersion is the standard deviation of the velocities. This study will find $\frac{V}{\sigma}$ for the merger remnant using the average velocity of all of the particles and their total velocity dispersion. The second question this study will address is: how does the velocity dispersion of the merger remnant depend on orbital radius? Does it peak in the center and decrease as for observed ellipticals (\cite{10.1093/mnras/stt442}), or does the velocity distribution follow a different trend? Both questions will be answered by analyzing baryonic matter, as that is what is observable in elliptical galaxies.\\

By answering these questions, we can begin to understand if mergers between galaxies such as the Milky Way and Andromeda can produce the ellipticals that are observed today.\\

This study aims to analyze parameters in a simulation that are directly observable in real galaxies. As a result, this will allow us to compare the results of what we believe form elliptical galaxies with actual galaxies. If the observables in the simulation closely match those of real elliptical galaxies, it will tell us that major mergers between large spiral galaxies can form ellipticals. If the results of the simulation do not match real elliptical galaxies, it will tell us that other mechanisms are required to form ellipticals.\\

\section{Methodology}

The simulation used in this study is from \cite{2012ApJ...753....8V} and simulates the Milky Way, Andromeda, M33 system from now until 11 gigayears (Gyrs) from now. It is an N-body simulation that integrates the positions and velocities of massive particles based upon the gravitational interactions from all other particles. Three particle types are used: disk, bulge, and dark matter halo particles. Finally, the simulation assumes a dry merger where there are no collisional interactions between particles, including gas drag.\\

This study will analyze the final snapshot of the simulation at 11 Gyrs to ensure the merger remnant between the Milky Way and Andromeda is as close to its final state as possible. In addition, the particles will be rotated into a new reference frame such that the net angular momentum vector of the remnant is parallel with the Z axis of a new coordinate system. Additionally, the center of mass of the remnant will be placed at the origin. These transformations will simplify the analysis of the remnant. Once the transformation is done, simple plots such as the Y velocity of particles as a function of X can be plotted as shown in Figure \ref{velocity} to determine if the remnant has a net rotation. Next, the average circular velocity of all disk and bulge particles will be calculated and divided by the standard deviation of the velocities to determine $\frac{V}{\sigma}$. Finally, the velocity dispersion will be determined at increasing radii from the center of the remnant and plotted as a function of radius. This will be done by calculating the velocity dispersion of particles within $\pm$dr of each given radius, where dr is 0.5 Kpc.\\

\begin{figure}
    \centering
    \plotone{velocity.png}
    \caption{Y velocity (into the page) as a function of X for the Milky Way -- Andromeda Merger remnant. Particles at negative X values appear to move in the negative Y direction while particles at positive X values appear to move in the positive Y direction, clearly showing rotation.}
    \label{velocity}
\end{figure}

The simulation provides the X, Y, and Z components of velocity for each particle. However, the circular velocity is required to find $\frac{V}{\sigma}$, and is taken as the absolute velocity in the $\hat{y}$ direction of the new coordinate system as shown in Equation \ref{velo_mag}:

\begin{equation}
    |V|=|v_y|
    \label{velo_mag}
\end{equation}

In addition, the velocity dispersion, $\sigma$ will be required. $\sigma$ is the standard deviation of the $\hat{y}$ velocities for all particles and is given by Equation \ref{std}:

\begin{equation}
    \sigma=\sqrt{\frac{\sum\limits_{n=1}^{N} (v_{y_n}-\mu)}{N}}
    \label{std}
\end{equation}

Where $\mu$ is the average $\hat{y}$ velocity.\\

Determining $\frac{V}{\sigma}$ will not require a plot. However, a plot of the velocity dispersion as a function of radius will be needed. This plot will then be compared to similar plots of elliptical galaxies to determine if they share the same characteristics, hence showing if a major galaxy merger can create a classical elliptical galaxy.\\

Because the Milky Way and Andromeda are inclined to each other, and because the merger in the simulation is collisionless, it is expected that the majority of the stars will orbit in random directions, resulting in little or no net rotation and a $\frac{V}{\sigma}<0.6$. This is typically observed in large ellipticals similar to the merger remnant. It is also predicted that the velocity dispersion will be greatest near the center of the remnant and decrease at larger radii because there are many more particles near the core, allowing for greater variations in velocity. In addition, the particle velocities are much higher near the core, so a small percentage deviation from the mean velocity results in a larger change in velocity. A higher dispersion near the core is also observed in many ellipticals.

\section{Results}

Figure \ref{merger_vsigma} shows the $\frac{V}{\sigma}$ of the merger remnant as a function of radius. $\frac{V}{\sigma}$ has a minimum value of approximately 0.75, and increases as a function of radius. The total average $\frac{V}{\sigma}$ is 0.8, indicating that the merger remnant is a fast rotator.\\

\begin{figure}
    \centering
    \plotone{vsigma.png}
    \caption{The velocity divided by velocity dispersion of the Milky Way-Andromeda merger remnant plotted as a function of distance from the merger remnant, in units of kpc. In general, $\frac{V}{\sigma}$ is smallest near the center of the remnant and grows at larger radii. $\frac{V}{\sigma}$ has a minimum value of 0.75, and thus the merger remnant is a fast rotator.}
    \label{merger_vsigma}
\end{figure}

Figure \ref{merger_disp} shows the velocity dispersion of the merger remnant as a function of radius. The velocity dispersion is greatest at the center of the remnant, and quickly drops at increasing radii.\\

\begin{figure}
    \centering
    \plotone{velo_disp.png}
    \caption{Velocity dispersion of the Milky Way-Andromeda merger remnant, in units of km s$^{-1}$, as a function of distance from the center of the remnant, in units of kpc. The velocity dispersion peaks at the center of the galaxy and quickly decreases with radius.}
    \label{merger_disp}
\end{figure}

\section{Discussion}

As mentioned in \S \ref{proposal}, a slow rotating galaxy has a $\frac{V}{\sigma}<0.6$. The merger remnant in this simulation has a $\frac{V}{\sigma}$=0.8, and is thus a fast rotator. This is unexpected because most observed elliptical galaxies are considered slow rotators. This has two possible implications. The first is that a collision between two large spiral galaxies cannot explain the ellipticals that are observed in the universe. The second is that the remnant may not have reached its final state by the time the simulation concluded. The merger occurred $\sim$4 Gyr into the simulation, and this work uses data from 11 Gyr into the simulation. While it would be expected for the remnant to settle 7 Gyr after the merger event, it is possible that the system had not yet reached dynamical stability by this time. Because of this, it is possible that running the simulation for a few more Gyrs would yield a result more similar to that of observed elliptical galaxies.\\

Figure \ref{merger_disp} shows that the velocity dispersion of the merger remnant peaks at its core and decreases at increasing distances from the center. This result is observed in elliptical galaxies, and indicates that a merger between two larger spiral galaxies can at least explain some of the properties of elliptical galaxies. Thus, this simulation suggests that the formation mechanism of elliptical galaxies likely involves mergers between spiral galaxies.\\

\section{Conclusion}

The simulated merger remnant of the Milky Way-Andromeda collision is probed for its kinematic properties, which are then compared to those of observed elliptical galaxies. This analysis deepens our understanding of the origins of elliptical galaxies and allows us to test our theories of their formation mechanisms. This study compared the rotation rate and velocity dispersion curve of the merger remnant to observed elliptical galaxies to understand if mergers between large spiral galaxies are likely candidates for the formation mechanism of elliptical galaxies.\\

The average velocity of the merger remnant divided by its velocity dispersion, $\frac{V}{\sigma}$, is 0.8. This is greater than the slow rotator threshold of 0.6. Most large elliptical galaxies are characteristically slow rotators, and thus this merger does not explain the slow rotation of most large elliptical galaxies. This is contrary to what was expected, and may have to do with the simulation ending too early, not allowing the merger remnant to settle into its final state. This may also suggest that mergers between two large spiral galaxies cannot fully explain observed elliptical galaxies.\\

The velocity dispersion of the merger remnant peaks at the core, and sharply decreases as a function of radius. This matches observed elliptical galaxies, indicating that mergers between spiral galaxies likely play a role in the formation of elliptical galaxies.\\

To further understand the formation mechanism of elliptical galaxies, the simulation itself could be run longer to allow the merger remnant to approach its final state. In addition, elliptical galaxies may be the result of several mergers of varying size, and thus simulating a galaxy that undergoes several mergers would further our knowledge of how ellipticals form.

\section{Acknowledgements}

I would like to thank Gurtina Besla for providing the base code for this project as well as the simulation data for the Milky Way-Andromeda merger. In addition, I thank Hayden Foote for providing assistance and feedback on my code. Finally, I thank Mika Lambert for her assistance with NumPy methods.

\software{NumPy \cite{harris2020array}, Astropy \cite{astropy:2022}, Matplotlib \cite{Hunter:2007}}

\bibliography{references.bib}{}
\bibliographystyle{aasjournal}

\end{document}
